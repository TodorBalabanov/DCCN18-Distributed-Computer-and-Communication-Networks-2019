\documentclass[11pt]{article}
\usepackage[english]{babel}
\usepackage{url}
\usepackage{graphicx,DCCN2019_en}

\pagestyle{fancy}
\fancyhead{} 
\fancyfoot{}

\usepackage[utf8]{inputenc}
\linespread{1.0}

\usepackage{amsmath}

\makeatletter
\fancyhead[RO]{\small DCCN 2019 \\ {23-27 September 2019}}
\fancyhead[LO]{\small Ketipov, Kostadinov, Petrov, Zankinski, Balabanov \\ Human-Computer MDC for TS Forecasting}
% Rumen Ketipov
% Georgi Kostadinov
% Plamen Petrov
% Iliyan Zankinski
% Todor Balabanov

\c@page=1 
       
\makeatother

\title{Human-Computer Mobile Distributed Computing for Time Series Forecasting}

\author[1]{\small R.R. Ktipov}
\author[1]{\small G.B. Kostadinov}
\author[1]{\small P.D. Petrov}
\author[1]{\small \\I.A. Zankinski}
\author[0000-0003-3139-069X]{\small T.D. Balabanov}

\affil[1]{\footnotesize Institute of Information and Communication Technologies \break Bulgarian Academy of Sciences \break acad. Georgi Bonchev Str., block 2, office 514, 1113 Sofia, Bulgaria}

\email{rketipov@iit.bas.bg, g.kostadinov@iit.bas.bg, p.petrov@iit.bas.bg, iliyan@hsi.iccs.bas.bg, todorb@iinf.bas.bg}

\begin{document}

\udc{004.93}

{\let\newpage\relax\maketitle}

\vskip -1.5em

\footnotetext{This work was supported with private funding by Velbazhd Software LLC.}

\begin{abstract}
Distributed computing became very popular in the last two decades. In many cases distributed computing projects are based on a donated calculation power. The most famous donated distributed computing project is SETI@home, which is related to deep space signals processing in attempt to find an alien life. The usage of donated distributed computing had its influence in the time series forecasting in the face of MoneyBee\cite{bohn01} project. MoneyBee project was a desktop screensaver application, which was calculating financial time series forecasting by training of artificial neural networks with evolutionary algorithms. With the expansion of the mobile devices in last decade it become relevant some donated distributed computing solutions to be developed as mobile applications. Such solution was developed at IICT-BAS\cite{tomov01}, which is based on Android Live Wallpaper technology. This research proposes an extension of the work done at IICT-BAS in the direction of human-computer based distributed computing by providing software capabilities of the users to vote for future financial changes. 

\keywords{distributed computing, time series forecasting, artificial neural networks, evolutionary algorithms}
\end{abstract}

\section{Introduction} \label{Introduction}

\section{Model Proposition} \label{Model Proposition}

\section{Experiments and Results} \label{Experiments and Results}

\section{Conclusion} \label{Conclusion}

\vskip 1.5em

\begin{thebibliography}{99}

\bibitem{bohn01} Bohn, A., Guting, T., Mansmann, T., \textbf{\textit{MoneyBee: A new product to predict stock market developments using artificial intelligence and increased calculation capacitiy}} (in German), et al. Wirtschaftsinf, vol. 45, no. 3, p. 325--333, 2003.

\bibitem{tomov01} Tomov, P., Zankinski, I., Barova, M., \textbf{\textit{Mobile Alternative of the MoneyBee Project for Financial Forecasting}}, Proceedings of the Annual University Scientific Conference of the National Military University Vasil Levski, Veliko Tarnovo, p. 1085--1089, 2018.

\end{thebibliography}

\end{document}
